\chapter{Conclusiones y Trabajo Futuro}
\label{cap:conclusiones}

Conclusiones del trabajo y líneas generales de trabajo futuro.

Teniendo encuenta el proceso completo, el desarrollo de VitHabitus ha supuesto un reto técnico e interesante al ser mi presentación al mundo de las aplicaciones móviles. 

Al final se ha conseguido cumplir gran parte de los objetivos propuestos, con el desarrollo del frontend y el backend con la API y la base de datos de Firebase.
Integración del algoritmo de recomendación de hábitos y un correcto flujo de funcionamiento entre todos los elementos.

Queda como resultado una aplicación que supone de gran utilidad para muchas personas y que demuestra una vez más, el increible potencial que tiene la inteligencia artificial aplicada a la salud. Además demuestra como el desarrollo de aplicaciones móviles puede actuar como un canal eficaz para trasladar tecnologías complejas a la vida cotidiana de las personas, gracias a una interfaz intuitiva y de fácil uso a través de un teléfono móvil.

\subsection{Trabajo Futuro}

Independientemente del resultado. Siempre hay posibilidad de mejora y avances. 
De esta forma, hay algunos ideas que surgieron durante la realización del proyecto.
Una de las más importantes sería la ampliación del formulario de hábitos, permitiendo ser aún más flexible con más hábitos que representen relación con la obesidad. 

\begin{itemize}
    \item Implementación de conectividad con dispositivos wearables que contienen sensores que podrían calcular datos como pulsaciones, horas de sueño, calorias quemadas, que pueden ser de gran utilidad para el recomendador.
    \item Publicar la aplicación de forma oficial en google play y App Store, lo que supondría cumplir con los estandares solicitados y mantener la aplicación actualizada
    \item Traducir la aplicación a varios idiomas o al menos al inglés para tener futuro en el mercado internacional. 
    \item Por otro lado en la parte visual, hay ciertos partes que se podrían mejorar para obtener un acabado más profesional.
\end{itemize}

. 



